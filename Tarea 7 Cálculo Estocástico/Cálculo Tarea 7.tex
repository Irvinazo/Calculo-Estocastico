\documentclass[letterpaper]{article} 
\usepackage[left = 0.5in, right = 0.5in, top = 0.9in, bottom = 0.9in]{geometry}
\usepackage{enumitem}
\usepackage{multicol}
\usepackage[spanish]{babel}
\usepackage[utf8]{inputenc}

\usepackage{amsmath,amssymb,amsthm}
\usepackage{tikz-cd}
\usepackage{mathrsfs}
\usepackage[bbgreekl]{mathbbol}
\usepackage{dsfont}
\usepackage{graphicx}
\graphicspath{{img/}}

\newcommand{\op}{\operatorname}
\newcommand{\Op}{^{\op{op}}}
\newcommand{\scc}{\mathscr C}
\newcommand{\scd}{\mathscr D}
\newcommand{\sce}{\mathscr E}
\newcommand{\sci}{\mathscr I}
\newcommand{\scj}{\mathscr J}
\newcommand{\scx}{\mathscr X}
\newcommand{\var}{\mathrm{Var}}
\newcommand{\Id}{\operatorname{Id}}
\newcommand{\N}{\mathbb N}
\newcommand{\Z}{\mathbb Z}
\newcommand{\Q}{\mathbb{Q}}
\newcommand{\I}{\mathbb{I}}
\newcommand{\R}{\mathbb{R}}
\newcommand{\C}{\mathbb{C}}
\newcommand{\F}{\mathcal{F}}
\newcommand{\G}{\mathcal{G}}
\newcommand{\B}{\mathcal{B}}
\newcommand{\abs}[1]{\left\lvert #1 \right\rvert}
\newcommand{\inv}{^{-1}}
\renewcommand{\to}{\rightarrow}
\newcommand{\ent}{\Longrightarrow}
\newcommand{\E}{\mathbb{E}}
\renewcommand{\P}{\mathbb{P}}
\newcommand{\1}{\mathds{1}}
\renewcommand{\qedsymbol}{$\blacksquare$}

\theoremstyle{definition}
\newtheorem{dfn}{Definición}
\theoremstyle{definition}
\newtheorem{teo}{Teorema}
\theoremstyle{definition}
\newtheorem{cor}{Corolario}
\theoremstyle{definition}
\newtheorem{prop}{Proposición}
\theoremstyle{definition}
\newtheorem{obs}{Observación}


\title{\textbf{Cálculo Estocástico\\
Tarea 7}}
\author{Iván Irving Rosas Domínguez}
\date{\today}

\DeclareSymbolFontAlphabet{\mathbbm}{bbold}
\DeclareSymbolFontAlphabet{\mathbb}{AMSb}
\DeclareMathSymbol\bbDelta  \mathord{bbold}{"01}

\begin{document}
\maketitle

%\begin{abstract}
%\end{abstract}

\begin{itemize}
    \item[\textbf{1.}] Demostrar el siguiente Teorema:\\
    
    \textbf{Teorema:}
    Sea $X(t)$ una solución de la ecuación 
    \[
        dX(t)=\mu(X(t))dt+\sigma(X(t))dB(t).  
    \]
    y definamos $Y(t)=X(\tau_t)$. Entonces $Y(t)$ es una solución débil a la ecuación
    diferencial estocástica 
    \[
        dY(t)=\frac{\mu(Y(t))}{g(Y(t))}dt+\frac{\sigma(Y(t))}{\sqrt{g(Y(t))}}dB(t), \qquad \text{ con } \quad Y(0)=X(0).   
    \]
    \item[\textbf{2.}] Si $X(t)$ es una difusión con coeficientes $\mu(x)=cx$ y $\sigma(x)=1$.
    Dar su generador y mostrar que 
    \[
    X^2(t)-2c\int_{0}^{t}X^2(s)ds-t    
    \]
    es una martingala.
    \item[\textbf{3.}] $X(t)$ es una difusión con $\mu(x)=2x$ y $\sigma^2(x)=4x$. Dar su generador 
    $L$. Resolver $Lf=0$ y dar una martingala $M_f$. Hallar la EDE para el proceso 
    $Y(t)=\sqrt{X(t)}$, y dar el generador de $Y(t)$.
    \item[\textbf{4.}] Mostrar que el tiempo medio de salida de una difusión a partir 
    de un intervalo, que por el teorema (\dots) satisface la ecuación diferencial 
    \[
        Lv=-1,
    \]
    está dada por 
    \[
    v(x)=-\int_{a}^{x}2G(y)\int_{a}^{y}\frac{ds}{\sigma^2(s)G(s)}dy+\int_{a}^{b}2G(y)\int_{a}^{y}\frac{ds}{\sigma^2(s)G(s)}dy\frac{\int_{a}^{x}G(s)ds}{\int_{a}^{b}G(s)ds},    
    \]
    donde $G(x)=\exp \left(-\int_{a}^{x}\frac{2\mu(s)}{\sigma^2(s)}ds\right)$.
    \item[\textbf{5.}] Dar una representación probabilística de la solución $f(x,t)$ de la EDP
    \[
        \frac{1}{2}\frac{\partial^2f}{\partial x^2}+\frac{\partial f}{\partial t}=0, \quad 0\leq t\leq T, \quad f(x,T)=x^2.    
    \]
    Resolver esta EDP utilizando la solución de la correspondiente ecuación diferencial estocástica.
\end{itemize}
\end{document}